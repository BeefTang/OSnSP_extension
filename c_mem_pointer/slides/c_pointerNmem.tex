\documentclass{beamer}
\usepackage{color,soul}
%Information to be included in the title page:
\title{OS n' SP extension with respect to C pointer \\{\small Halloween's Episode--- No treats, Only tricks}}
\author{E}
\institute{University of Birmingham}
\date{2023 Winter}

\begin{document}
\frame{\titlepage}

\AtBeginSection[]
{
  \begin{frame}
    \frametitle{Table of Contents}
    \tableofcontents[currentsection]
  \end{frame}
}
\begin{frame}
    \frametitle{Table of Contents}
    \tableofcontents
\end{frame}


\section{Pointer, array and address}
\begin{frame}{What address is?}
  There is $2^m byte$ of memory, so m bits can represent the whole memory space. 
  Every resident/cell is one byte. 
  Normally 8 bytes of pointer---8*8=64 bits. $2^64=16 EiB$
  "8 byte" is so intrinsic that it has a special name, doubleword. 4 bytes --- word
\end{frame}

\begin{frame}{What array is}
  A consequtive memory space.
  The name is synonymous to the address of the first element of the array.
\end{frame}

\begin{frame}{A concept: L-value}
  \begin{itemize}
    \item <1->   
    int a; // a has address \&a  \\
    int *p;// p has address \&p;
    \item <2->
    a and p, we name as L-value:\\
    A L-value is an object that occupies some identifiable location in memory (Legal operand of \&).
    \item<3-> int arr[4];//Is arr a  L-value?
    \item <4-> \&arr; //This operation is definitely LEGAL
  \end{itemize}
  
\end{frame}

\begin{frame}{What array is}
  \begin{itemize}
    \item <1->   
    arr[0]==*arr; \&arr[0]==a;\\
    What is \&arr? Try!
    \item <2->
    Assignment to arr is illegal, unlike ordinary variables. 
    \item<3-> error: assignment to expression with \textbf{array type}
    \item <4-> Meanwhile, it returns address(p=array;), but its address is again the same address.
    \item <5-> Try some arithmetic, like +1, comparing \&arr+1 and arr+1.
  \end{itemize}
 
\end{frame}

\begin{frame}{What array is from compiler's point of view}
  "array" is not a pointer but reduced to pointer:
  \begin{quote}
    Except when it is the operand of the sizeof operator, the \_Alignof operator, or the unary \& operator, or is a string literal used to initialize an array, an expression that has type "array of type" is converted to an expression with type "pointer to type" that points to the initial element of the array object and is \textbf{not an lvalue}. If the array object has register storage class, the behavior is undefined.
    \\     ---C11 6.3.2.1
  \end{quote}
\end{frame}

\begin{frame}{What pointer is?}
A variable which has an address as value. n bits in n system, and n is elementary unit on which CPU do computation.

%The syntax of the declaration for a variable mimics the syntax of expressions in which the variable might appear. The same idea idea also applies to function.(they are in the both "environment")
\end{frame}

\begin{frame}{Pointer and function: As parameters and references}
  \begin{itemize}
    \item <1-> C can only pass by value
    \item <2-> It is why scanf needs pointer as parameter
    \item <3-> We can also have pointer to function 
    \item <4-> Then pointer array to functions
    \item <5-> Then pointer to function as parameter (for a general function)
  \end{itemize}
\end{frame}

\begin{frame}{Pointer Or Array as formal parameter}
fun(char a[]) and fun (char *a) is identical, not just synonymous.
Try some arithmetic on a.
\end{frame}

\begin{frame}{Mulidimensional array}

\end{frame}

\begin{frame}{Some features}
  Variable length array.
  
  f(const a[])

  int sum2d(int rows, int cols, int ar[rows][cols])

\end{frame}

\section{When related to memory}

int a;
int b[3];
int c; 

b[-1]=0;

\section{Useful tricks}
%#define A(r, c) (A[(r)*WIDTH + (c)])

flexible struct


\end{document}