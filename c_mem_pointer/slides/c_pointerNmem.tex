\documentclass{beamer}
%Information to be included in the title page:
\title{OS n' SP extension with respect to C pointer and memmory}
\author{E}
\institute{University of Birmingham}
\date{2023 Winter}

\begin{document}
\frame{\titlepage}

\AtBeginSection[]
{
  \begin{frame}
    \frametitle{Table of Contents}
    \tableofcontents[currentsection]
  \end{frame}
}
\begin{frame}
    \frametitle{Table of Contents}
    \tableofcontents
\end{frame}


\section{Pointer, array and address}
%K&R 5.1-5.5
\begin{frame}{What address is?}
  $2^m bytea$, so m bits can represent the memory space, every resident is one byte
\end{frame}

\begin{frame}{What array is}
  A consequtive memory space.
  Defining a array, the name refers to the forst element of the array.
  %a[0]==*a; &a[0]==a;

  a=... //not allowed, for a is not a lvalue
  lvalue is 
  Array is not pointer but reduced to pointer.

\end{frame}

\begin{frame}{What point is?}
A variable which has address as value. n bits in n system, and n is elementary unit on which CPU do computation.
\end{frame}

\begin{frame}

a=... //not allowed, for a is not a lvalue
  lvalue is 
  Array is not pointer but reduced to pointer.
  
\end{frame}


\section{Compound pointer}

\section{Functionality via pointer}

\end{document}